Помимо рабочего движения, в Швейцарии существует и
социалистическое женское движение. Издается двухнедельный
журнал «Застрельщица». Социалистки разделяются на группы,
которые по своим организационным формам сходны совершенно с
партийной социалистической организацией. Тяжелые
экономические условия во время войны, промышленный кризис,
который уже в течение нескольких лет надвигается, и не
прекращающееся вздорожание жизни пробудили женщину из
инертного состояния и толкнули ее в ряды профессиональных и
политических организаций. В последних громадных, массовых
демонстрациях и забастовках женщины-пролетарии играли роль
фермента, который двигал и толкал на дальнейшую борьбу.
Особенно в больших городах происходили грандиозные женские
демонстрации против дороговизны, которые заканчивались
порой нападением на чиновников и разгромом лавок.